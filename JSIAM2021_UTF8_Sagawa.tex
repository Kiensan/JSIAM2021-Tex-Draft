% --------------------------------------------------------------------------------------------
% ---------- 日本応用数理学会年会予稿集原稿テンプレート ----------
% --------------------------------------------------------------------------------------------
%
% ---------- 変更しないでください(ここから) ----------
\documentclass[11pt,a4j]{jarticle}
\setlength{\oddsidemargin}{0cm}
\setlength{\topmargin}{-.5cm}
\setlength{\headheight}{0cm}
\setlength{\headsep}{0cm}
\setlength{\footskip}{0cm}
\setlength{\textwidth}{16cm}
\setlength{\textheight}{24.7cm}
\setlength{\abovecaptionskip}{0cm}
\setlength{\footskip}{1cm}
\usepackage{fancyhdr}
\pagestyle{fancy}
\renewcommand{\headrulewidth}{0.0pt}
\lhead{}
\chead{}
\rhead{}
\renewcommand{\footrulewidth}{0.4pt}
\lfoot{\footnotesize 日本応用数理学会 2021年度 年会 講演予稿集 (2021.9.7--9)  Copyright (C) 2021 一般社団法人日本応用数理学会}
\cfoot{}
\rfoot{}
\makeatletter
\renewcommand{\baselinestretch}{1.3}\selectfont
\def\Title#1{{\Large\bf#1}\\[6pt]}
\def\Author#1{{\normalsize\hspace*{2zw}#1}\\[-4pt]}
\def\Affiliation#1{{\normalsize\hspace*{2zw}#1}\\[-5pt]}
\def\Email#1{\hspace*{2zw}e-mail : #1\\[0pt]}
\renewcommand{\section}{\@startsection{section}{1}{\z@}%
{2ex}{1ex}{\reset@font\large\bfseries}}%
\renewcommand{\thesection}{\@arabic\c@section}
\def\@listi{\topsep=.3\baselineskip \parsep=.2ex \partopsep=0ex%
\itemsep=0ex \leftmargin=4ex \rightmargin=2ex}
\let\@listI\@listi
\@listi\def\@listii{\parsep=.2ex \partopsep=0pt \itemsep=0ex%
\leftmargin=4ex \rightmargin=0ex}
\let\@listiii\@listii
\let\@listiv\@listii
\let\@listv\@listii
\let\@listvi\@listii
\long\def\@makecaption#1#2{\footnotesize\sbox\@tempboxa{#1. #2}
\ifdim\wd\@tempboxa >\hsize #1. #2\par
\else \global\@minipagefalse
\hb@xt@\hsize{\hfil\box\@tempboxa\hfil}
\fi}
\makeatother
% ---------- 変更しないでください(ここまで) ----------
%
% ---------- お好みで変更および追加してください(ここから) ----------
\usepackage{graphicx}
\usepackage{amsmath}
\usepackage{url}
\newtheorem{thm}{定理}
\newtheorem{df}[thm]{定義}
\newtheorem{lem}[thm]{補助定理}
\newtheorem{prop}[thm]{補題}
\def\proof{{\bf 証明}\hspace*{1zw}}
\def\thanks{~\\[.5\baselineskip]{\bf 謝辞}\hspace*{1zw}}
\def\labelenumi{\theenumi)}
%  ------自前で追加--------
\usepackage{amssymb}
\usepackage{listings}
\usepackage{algorithm,algorithmic}
 \usepackage{comment}
% listingsの設定
\lstset{
	columns=fullflexible,
	basicstyle=\ttfamily,
	xleftmargin=0.1cm, % left margin
	aboveskip=1pt,  % top margin
	belowskip=1pt % bottom margin
}
% 斜体のまま太字にする(大文字・小文字両方可)
\newcommand{\bm}[1]{{\text{\boldmath $#1$}}}
% 演算子と変数の定義
\DeclareMathOperator{\odivide}{\circ \hspace{-.50em} \backslash}
\DeclareMathOperator{\spt}{\,|\,}
\newcommand{\veps}{\varepsilon}
\newcommand{\hveps}{\bar{\varepsilon}}
%
% ---------- お好みで変更および追加してください(ここまで) ----------
%
\begin{document}
\twocolumn[
%
% ---------- タイトル,著者名,所属,e-mailアドレスを記入してください(ここから) ----------
\Title{Haskellを用いたMin-max-plusスケジューリングの求解}
%%% 著者,所属が複数行になる場合は \author{},\affiliation{}を追加してください
\Author{佐川 恭平$^{1}$, 五島 洋行$^{2}$}
\Affiliation{$^{1}$長岡技術科学大学 情報経営システム工学課程,$^{2}$法政大学 理工学部 経営システム工学科}
\Email{kyohei.sagwa@gmail.com}
% ---------- タイトル,著者名,所属,e-mailアドレスを記入してください(ここまで) ----------
%
]
% ---------- 変更しないでください(ここから) ----------
\renewcommand{\baselinestretch}{0.95}\selectfont
% ---------- 変更しないでください(ここまで) ----------
%
% ---------- 本文(ここから)--------------------------------------------------------------------------------------------
%
% ---------------------------------------
\section{はじめに}
% ---------------------------------------
Program evaluation and review technique (PERT)のスケジューリング問題に対して、Max-plus代数を利用したアプローチがある[1]。PERTにおける最早完了時刻はMax-plus線形システム(MPL)で表されKleene閉包と呼ばれる演算によって代数的に解を求めることが可能である。

一方、スケジューリング問題の代数的解法には、取り扱いが簡単になることと引き換えに計算量が多いことが問題になる。そこで工程の順序関係に従ってトポロジカルソートを行うことによって効率的に求解する方法が存在する[2]。両者の方法は代数的な明快さと計算量のトレードオフとなっておりそれら両方を満たす効率良い解法が存在しなかった。

MPLではスケジューリングするネットワークが構造依存となる。CPUスケジューリングで知られるFirst come first served (FCFS)では最も早く到着したジョブを工程の開始時間とするため、min演算が現れるが、Max-plus代数では取り扱うことができない。この問題を解決するためにはmax演算とmin演算が混合した記述が可能Min-max-plus system (MMPS) [3]を導入する必要がある。しかしMMPS上で表されるスケジューリング問題に対して解析解を求めるアルゴリズムは存在しない。そこで本研究では、MMPS上のMin-max-plus方程式で定式化されるスケジューリング問題について、Haskellを利用した代数表現による求解を提案する。
%
% ---------------------------------------
\section{MMPSによるスケジューリング問題の定式化}
% ---------------------------------------
Max-plus代数は、$\mathbb{R}_{\rm{max}}=\mathbb{R} \cup \{ \veps \}$上で演算$x \oplus y \equiv \rm{max}$$(x, y)$と$x \otimes y \equiv x + y$が定義された代数系である。さらにMin-max-plus代数では演算$x \wedge y = \rm{min}(x,y)$と$x \odivide y = -x + y$を追加する。ここで台集合の元$\veps$は通常$-\infty$で定義されるが本研究ではHaskellでの取り扱いを簡素にするため、
%
\begin{eqnarray}
x \oplus \veps &=& \veps \oplus x = x \wedge \veps = \veps \wedge x = x , \nonumber \\
x \otimes \veps &=& \veps \otimes x = \veps \,\,\,(x \in \mathbb{R}_{\rm{max}}) \nonumber
\end{eqnarray}
%
を満たす元と定義する。

Min-max-plus代数で定義されるMin-Max-plus System(MMPS)のMMPS表現$f$は
\begin{equation}
f = x_i \spt \alpha \spt f_k \oplus f_l \spt f_k \wedge f_l \spt f_k \otimes f_l
\end{equation}
と再帰的に表現される。ここで$x_i$は変数、$\alpha \in \mathbb{R}$、$f_k , f_l$はMMPS表現である。
%

MMPSでの時刻スケジューリングを定式する。ある工程$i$は作業時間$p_i$を持ち、先行工程からの入力が複数の場合は、先行工程からの入力が1番早いものを入力として作業を開始する$\rm{min}$工程と1番遅いものを入力として作業を開始する$\rm{max}$工程がある。ここで行列
%
\[
[\bm{F}_+]_{ij} = \begin{cases}
\text{$e$:工程iが先行工程jをもち} \\
\,\,\, \text{かつ工程$i$が$\rm{min}$工程ではない} \\
\text{$\veps$:それ以外} \\
\end{cases}
\]
%
\[
[\bm{F}_-]_{ij} = \begin{cases}
\text{$e$:工程iが先行工程jをもち} \\
\,\,\, \text{かつ工程$i$が$\rm{max}$工程ではない} \\
\text{$\veps$:それ以外} \\
\end{cases}
\]
%
\[
[\bm{u}]_i = \begin{cases}
\text{$u_i$:工程$i$が入力$u_i$を持つ} \\
\text{$\veps$:それ以外}
\end{cases}
\]
%
を定義し、工程の総数を$n$とすると工程の作業完了時刻$\bm{x} = [ x_1 \, \cdots \, x_n ]^T$は
\begin{equation}
\bm{x} = \bm{P} \otimes \{ \bm{F}_+ \otimes \bm{x} \oplus \bm{F}_- \odivide \bm{x} \oplus \bm{u} \}
\end{equation}
となる。ここで$\bm{P} = \rm{diag}$$[p_1 \, \cdots \, p_n]$である。

MPLの場合$\bm{x} = \bm{A} \otimes \bm{x} \oplus \bm{b}$となり、右辺の$\bm{x}$に繰り返し代入することにより解を求めることができるが、MMPSの場合は陽的に解をもとめることができない。しかしHaskellを導入することによりネットワーク構造がDAGの場合には解を求めることが可能である。
%
% ---------------------------------------
\section{HaskellによるMin-max-plusの表現}
% ---------------------------------------
Haskellの機能である型クラスとデータ型宣言を利用して行列上でMMPSの計算を可能とすること目標とする。Haskellではデータ構造をプログラミング言語の型として表現する。ここで型クラスを導入することにより、任意の型に制約を加えることが可能となる。そこでまず、Min-max-plus代数はHaskellの型クラスを利用して
\begin{lstlisting}
class MinMaxPlus a
	zero :: a
	one :: a
	oplus :: a -> a -> a
	otimes :: a -> a -> a
	wedge :: a -> a -> a
\end{lstlisting}
と記述される。このプログラムは、任意の型aに対して、零元と単位元、そしてそれぞれMin-max-plusの演算を定義することによってMin-max-plus代数として表すことができることを意味している。
%
次にMMPS表現はHaskellのデータ型宣言によって
\begin{lstlisting}
data MMPS = Min MMPS MMPS 
	| Max MMPS MMPS | Eps
	| Min MMPS MMPS
	| Plus MMPS MMPS
	| Divide MMPS MMPS
	| Just Int
	| Variable Int
\end{lstlisting}
と記述される。この記述はHaskell上でMMPSを2分木によって保持することを可能とする。そして今定義したMMPSに対して適切な零元、単位元と演算を作成し、この定義をMinMaxPlus型クラスに適用する。
また行列を
\begin{lstlisting}
data (MinMaxPlus a) => Matrix a
	= Scalar a
	| Matrix [[a]]
\end{lstlisting}
と各要素にMMPS表現を持つ行列と定義する。そして同じくMinMaxPlus型クラスに適用する。これにより式(2)をHaskell上で計算することが可能となる。
%
\section{MMP方程式の求解アルゴリズム}
Haskell上でMMPS表現は2分木で表されていることから、グラフ$\bm{G}=(\bm{V}, \bm{E})$で表し$\bm{V} =  \bm{Op} $ $ \spt $ $ \mathbb{R}_{\rm{max}} $$ \spt \bm{Var}$、$\bm{Op} = \{ \oplus, \wedge, \otimes, \odivide \}$、$\bm{Var}=\text{変数の集合}$として数式上で表現する。

MMPSから再帰的に計算を行い値を返す関数$redex:\bm{G} \rightarrow \mathbb{R}_{\rm{max}}$を定義する。木の根ノードを取り出す関数$root:\bm{G} \rightarrow \bm{V}$と、
木の左右の子ノードを根ノードとし木を取り出す関数$child$$:\bm{G} \rightarrow (\bm{G}, \bm{G})$を補助関数として定義すると関数$redex$は
%
\begin{lstlisting}[escapechar=\@]
@$redex(x \spt root(x) \in \mathbb{R}_{\rm{max}}) = root(x)$@
@$redex(x \spt root(x) \in \bm{Op}) =$@
@$\;\;\; redex(l) \; root(x) \; redex(r) \spt (l, r) = child(x)$@
@$redex(x \spt root(x) \in \bm{Var}) = compute(root(x))$@
\end{lstlisting}
%
となる。
次にHaskell上で式(2)を計算した各作業完了時刻$x_i$を保持する連想配列$var:\bm{Var}$ $\rightarrow$ $\bm{G}$と最終的な値を保持する連想配列$memo:$ $\bm{Var} \rightarrow \mathbb{R}_{\rm{max}}$を用意すると、変数から値を求める関数$compute:\bm{Var} \rightarrow \mathbb{R}_{\rm{max}}$は
%
\begin{lstlisting}[escapechar=\@]
@$compute(x \spt memo[x] \neq null ) = memo[x]$@
@$compute(x \spt memo[x] = null)$@
@$\;\;\;\; = memo[x] \spt memo[x] \leftarrow redex(var[x])$@
\end{lstlisting}
%
となる。
%関数$redex$と$compute$は再帰関数となっており、前者は式の2分木を深さ優先探索(DES)で計算する関数、後者はネットワークをDESで計算する関数となる。数式はHaskellのデータ型宣言MMPSによりその構造が有向非サイクルグラフ(DAG)となることを保証するが、ネットワークに対してはDAGであることは保証されない。そこで、スケジューリングするシステムのネットワーク構造をDAGであると仮定する。
関数$redex$と$compute$は再帰関数となっているが、関数$compute$はネットワーク構造が有向非サイクルグラフ(DAG)であることを仮定することにより基底を保証する。

2つの関数を定義することにより各最早時刻を求めることができた。最終的に全ての最早完了時刻の連想配列$\overline{\bm{x}}$を求めるアルゴリズムは以下のようになる。
\begin{algorithm}
\caption{Min-max-plus方程式の再帰求解}
\begin{algorithmic}[1]
	\FOR {$x \leftarrow var$}
		\STATE $\overline{\bm{x}} \leftarrow compute(x)$
	\ENDFOR
\RETURN $\overline{\bm{x}}$ 
\end{algorithmic} 
\end{algorithm}
%
% ---------------------------------
\begin{thebibliography}{3}
% ---------------------------------
%
\bibitem{1}
五島洋行, Max-Plus代数を用いて日程計画問題を考える, 計測と制御, 第52巻 (2013), 1083--1089.
%
\bibitem{2}
Hiroyuki Goto, A Fast Computation for the State Vector in a Max-Plus Algebraic System with an Adjacency Matrix of a Directed Acyclic Graph, SICE Journal of Control, Measurement, and System Integration, Vol 4 No.5 (2011), 361--364.
%
\bibitem{3} 
B. De Schutter, T. J. J. van den Boom, Model predictive control for max-min-plus scaling systems, Proc. 2001 American Control Conf. (2001), 319--324.
%
\end{thebibliography}
% ---------- 本文(ここまで)--------------------------------------------------------------------------------------------
%
\end{document}
